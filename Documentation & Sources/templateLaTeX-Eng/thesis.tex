%%%%%%%%%%%%%%%%%%%%%%%%%%%%%%%%%%%%%%%%%%%%%%%%%%%%%%%%%%%%%%%%%%%%%%%%%%%%%
%%%
%%% File: thesis.tex, version 1.9, May 2016
%%%
%%% =============================================
%%% This file contains a template that can be used with the package
%%% cs.sty and LaTeX2e to produce a thesis that meets the requirements
%%% of the Computer Science Department from the Technical University of Cluj-Napoca
%%%%%%%%%%%%%%%%%%%%%%%%%%%%%%%%%%%%%%%%%%%%%%%%%%%%%%%%%%%%%%%%%%%%%%%%%%%%%

\documentclass[12pt,a4paper,twoside]{report}         
\usepackage{cs}              
\usepackage{times}
\usepackage{graphicx}
\usepackage{latexsym}
\usepackage{amsmath,amsbsy}
\usepackage{amssymb}
\usepackage[matrix,arrow]{xy}
\usepackage[T1]{fontenc}
\usepackage{ae,aecompl}
%\usepackage{shortcut} %definitii pentru diacritice; 
\usepackage{amstext}
\usepackage{graphics}
\usepackage[T1]{fontenc}
\usepackage{ae,aecompl}
\usepackage{algorithm}
%\usepackage{algorithmic}
\usepackage{color}
\usepackage{color}
\usepackage[super]{nth}
% \mastersthesis
\diplomathesis
% \leftchapter
\centerchapter
% \rightchapter
\singlespace
% \oneandhalfspace
% \doublespace

\renewcommand{\thesisauthor}{Mihai PÎȚU}    %% Your name.
\renewcommand{\thesismonth}{September}     %% Your month of graduation.
\renewcommand{\thesisyear}{2019}      %% Your year of graduation.
\renewcommand{\thesistitle}{SueC - An Editor and Interpreter for Pseudocode} 
\renewcommand{\thesissupervisor}{dr. eng. Emil Ștefan CHIFU}
\newcommand{\department}{\bf FACULTY OF AUTOMATION AND COMPUTER SCIENCE\\
COMPUTER SCIENCE DEPARTMENT}
\newcommand{\thesis}{LUCRARE DE LICEN'T'A}
\newcommand{\utcnlogo}{\includegraphics[width=15cm]{img/tucn.jpg}}

\newcommand{\uline}[1]{\rule[0pt]{#1}{0.4pt}}
%\renewcommand{\thesisdedication}{P\u{a}rin\c{t}ilor mei}

\begin{document}
%\frontmatter
%\pagestyle{headings}

\newenvironment{definition}[1][Defini\c{t}ie.]{\begin{trivlist}
\item[\hskip \labelsep {\bfseries #1}]}{\end{trivlist}}



%\thesistitle                    %% Generate the title page.
%\authordeclarationpage                %% Generate the declaration page.

\pagenumbering{arabic}
\setcounter{page}{4}



\begin{center}
\utcnlogo

\department

\vspace{4cm}

{\bf \thesistitle} %LICENSE THESIS TITLE}

\vspace{1.5cm}

LICENSE THESIS

\vspace{6cm}

Graduate: {\bf \thesisauthor} 

Supervisor: {\bf \thesissupervisor}

\vspace{3cm}
{\bf \thesisyear}
\end{center}

\thispagestyle{empty}
\newpage

\begin{center}
\utcnlogo

\department

\end{center}
\vspace{0.5cm}

%\begin{small}
\begin{tabular}{p{7cm}p{8cm}}
 %\hspace{-1cm}& APPROVED,\\
 \hspace{-1cm}DEAN, & HEAD OF DEPARTMENT,\\
 \hspace{-1cm}{\bf Prof. dr. eng. Liviu MICLEA} & {\bf Prof. dr. eng. Rodica POTOLEA}\\  
\end{tabular}
 
\vspace{2cm}

\begin{center}
Graduate: {\bf \thesisauthor}

\vspace{1cm}

{\bf \thesistitle}
\end{center}

\vspace{1cm}

\begin{enumerate}
 \item {\bf Project proposal:} {\it Short description of the license thesis and initial data}
\item {\bf Project contents:} {\it (enumerate the main component parts) Presentation page, advisor's evaluation, title of chapter 1, title of chapter 2, ..., title of chapter n, bibliography, appendices.}
\item {\bf Place of documentation:} {\it Example}: Technical University of Cluj-Napoca, Computer Science Department
\item {\bf Consultants:}
\item {\bf Date of issue of the proposal:} November 1, 2017
\item {\bf Date of  delivery:} February 18, 2019 {\it (the date when the document is submitted)}
  \end{enumerate}
\vspace{1.2cm}

\hspace{6cm} Graduate: \uline{6cm} 

\vspace{0.5cm}
\hspace{6cm} Supervisor: \uline{6cm} 
%\end{small}

\thispagestyle{empty}


\newpage
$ $
%\begin{center}
%\utcnlogo

%\department
%\end{center}

\thispagestyle{empty}
\newpage

\begin{center}
\utcnlogo

\department
\end{center}

\vspace{0.5cm}

\begin{center}
{\bf
Declara\c{t}ie pe proprie r\u{a}spundere privind\\ 
autenticitatea lucr\u{a}rii de licen\c{t}\u{a}}
\end{center}
\vspace{1cm}



Subsemnatul(a) \\
\uline{14.8cm}, 
legitimat(\u{a}) cu \uline{4cm} seria \uline{3cm} nr. \uline{4cm}\\
CNP \uline{9cm}, autorul lucr\u{a}rii \uline{2.8cm}\\
\uline{16cm}\\
\uline{16cm}\\
elaborat\u{a} \^{\i}n vederea sus\c{t}inerii examenului de finalizare a studiilor de licen\c{t}\u{a} la Facultatea de Automatic\u{a} \c{s}i Calculatoare, Specializarea \uline{7cm} din cadrul Universit\u{a}\c{t}ii Tehnice din Cluj-Napoca, sesiunea \uline{4cm} a anului universitar \uline{3cm}, declar pe proprie r\u{a}spundere, c\u{a} aceast\u{a} lucrare este rezultatul propriei activit\u{a}\c{t}i intelectuale, pe baza cercet\u{a}rilor mele \c{s}i pe baza informa\c{t}iilor ob\c{t}inute din surse care au fost citate, \^{\i}n textul lucr\u{a}rii \c{s}i \^{\i}n bibliografie.

Declar, c\u{a} aceast\u{a} lucrare nu con\c{t}ine por\c{t}iuni plagiate, iar sursele bibliografice au fost folosite cu 
respectarea legisla\c{t}iei rom\^{a}ne \c{s}i a conven\c{t}iilor interna\c{t}ionale privind drepturile de autor.

Declar, de asemenea, c\u{a} aceast\u{a} lucrare nu a mai fost prezentat\u{a} \^{\i}n fa\c{t}a unei alte comisii de examen de licen\c{t}\u{a}.

\^{I}n cazul constat\u{a}rii ulterioare a unor declara\c{t}ii false, voi suporta sanc\c{t}iunile administrative, respectiv, \emph{anularea examenului de licen\c{t}\u{a}}.

\vspace{1.5cm}

Data \hspace{8cm} Nume, Prenume

\vspace{0.5cm}

\uline{3cm} \hspace{5cm} \uline{5cm}

\vspace{0.5cm}
\hspace{9.4cm}Semn\u{a}tura

\thispagestyle{empty}

\newpage


%\listoftables
%\listoffigures

%\clearpage 
%\newpage

%\begin{comment}
\include{guideline} 
%\end{comment}

\newpage

\tableofcontents
\newpage

\chapter{Introduction - Project Context}
\pagestyle{headings}
\section{Project Context}

	Computer science and programming is taught in schools around Romania for at least 30 years, especially in high schools, but also in secondary schools, starting with the \nth{5} grade. Before introducing directly to a programming language, many teachers use a pseudocode language which serves as a mean of understanding programming concepts in a more universal manner, bringing it closer to the natural spoken language. I have decided to make an implementation of this pseudocode by creating an editor and interpreter for it.
	
	The purpose of this project is to create an easier way of learning programming concepts for students who are new into this domain. This will serve as a fresh renewal of software used in schools today, as older tools such as Code::Blocks and/or Free Pascal are still used in schools and programming contests.
 
	SueC is the name of the editor which creates, edits and compiles files which represent pseudocode files. This editor will work also like any other editors, providing some error-checking mechanisms and returning the result after compiling a pseudocode file.
	
\section{Motivation}
	During high school, many of my colleagues have struggled learning programming and computer science as they had issues in understanding the simple paradigms because of C programming language. They have improved throughout the high school due to the teacher using pseudocode as a mean of explaining simple algorithms and paradigms, but there were some struggle shown for some when changing the pseudocode into implementations in C. 
	
	Nowadays, this issue is still persistent in schools in Romania as C and Pascal are used as main programming languages for teaching, exams and computer science contests. There are some more interactive programming languages such as Scratch which uses a graphical interface for implementing simple programs, but since the target audience is for primary school students, there is a need for an attractive way of making secondary and high school students for understanding programming at their age group. 
	
	At the moment, there are platforms for learning code such as CodeCademy and Udemy which contain basic courses for people at every age, but the main focus is for people who have a little background in programming and computer science.
	
	

\chapter{Project Objectives and Specifications}


As the title of the project suggests - "An Editor and Interpreter for Pseudocode" - this is an application which will serve as an educational tool for using the pseudocode as a programming language.

For the users of this application(students and/or teachers), the main functionalities of this application are:
\begin{itemize}
	\item Creating/opened a file in which pseudocode can be implemented.
	\item Writing pseudocode in the file created/opened.
	\item Compiling the file and obtaining the desired result or error(s) if there are occured.
	\item Running some step-by-step basic tutorials which are aimed for learning the language.
\end{itemize}


The main objectives of this project are:
\begin{itemize}
	\item Developing an user-friendly application which handles the main file handling operations and communicating with the compiler of the pseudocode source files.
	\item Creating an understandable programming language that resembles the pseudocode used by teachers in schools and/or universities. For a technical point of view, the pseudocode will be created like any other programming languages, having similar elements to existing ones that are used nowadays, but also with specific structural elements bringing it closer to the natural language.
	\item Developing a compiler for this programming language by defining a lexical and syntactic analyzer respectively. These analyzers contain the set of rules that apply to the programming language. 
\end{itemize}





\chapter{Bibliographic research}

For this project, my research done was focused on the main components and technologies included in the project:
\begin{enumerate}\bfseries
	\item C Programming Language
	\item Python Programming Language
	\item Lex
	\item Yacc
\end{enumerate}
 
\section{C Programming Language}
 C is a general-purpose, procedural computer programming language supporting structured programming, lexical variable scope, and recursion, while a static type system prevents unintended operations. This programming language was created between 1972 and 1973 as a way of making utilities work in Unix operating system, later being used for reimplementing the kernel of this OS. Since 1980s, C has gained enough popularity becoming one of the most widely used programming languages in the world. During this time, there were several C compilers created by several vendors for being available for the majority of existing computer architectures and operating systems. Since 1989, C has been standardized by ANSI (American National Standards Institute) and by the International Organization for Standardization (ISO). 
  
 Being an imperative procedural language, C was designed to be compiled using a relatively straightforward compiler to provide low-level access to memory and language constructs that map efficiently to machine code instructios all with minimal runtime support. This language supports cross-platform programming, making it available in numerous platforms, from embedded microcontrollers and supercomputers. It also stood as a big influence in the creation of other programming languages, such as:
 \begin{itemize}
 	\item C++
 	\item C\#
 	\item Java
 	\item Python
 	\item Go
 \end{itemize}

 The C programming language syntax is defined by a formal grammar, having specific keywords and rules based on statements to specify different actions. The most common statement is an expression statement, consisting of an expression to be evaluated followed by a semicolon. The main structure of a C program consists of declarations and function definitions, which in turn contain declarations and statements. 
 
 Besides exxpressions, the main sequence execution of statements can contain several control-flow statements defined by reserved keywords:
  
 \begin{itemize}
 	\item Conditional execution 
 	
 		This is defined by \textit{if} and \textit{else} statements. These statements contain an expression that the \textit{if} checks if it is true or not and execute statements based on a condition.
 		
 		Alongside those statements, there exists the \textit{switch} statement in which it displays a \textit{case} based on the expression given.
 		
 	\item Iterative execution (Looping)
 	
 		This is defined by \textit{while} , \textit{do-while} and \textit{for} statements which can loop through a certain set. The \textit{for} statement contains separate expressions for initialization, testing and reinitialization, any of which can be omitted. 
 \end{itemize}


\section{Python Programming Language}

Python is an interpreted, high-level, general-purpose programming language with the aim to help programmers with clear, logical code for small and large-scale projects. This programming language is dynamically typed and garbage-collected, supporting multiple programming paradigms, such as: procedural, object-oriented and functional. Due to this and its comprehensive standard library, Python is often described as a "batteries included" language.



\chapter{Analysis and Theoretical Foundation}
\label{ch:analysis}

Together with the next chapter takes about 60\% of the whole paper

The purpose of this chapter is to explain the operating principles of the implemented application.
Here you write about your solution from a theory standpoint - i.e. you explain it and you demonstrate its theoretical properties/value, e.g.:
\begin{itemize}
 \item used or proposed algorithms
 \item used protocols
 \item abstract models
 \item logic explanations/arguments concerning the chosen solution
 \item logic and functional structure of the application, etc.
\end{itemize}

{\color{red} YOU DO NOT write about implementation.

YOU DO NOT copy/paste info on technologies from various sources and others alike, which do not pertain to your project.
}

\section{Title}
\section{Other title}


\chapter{Detailed Design and Implementation}

Together with the previous chapter takes about 60\% of the paper.

The purpose of this chapter is to document the developed application such a way that it can be maintained and developed later. A reader should be able (from what you have written here) to identify the main functions of the application.

The chapter should contain (but not limited to):
\begin{itemize}
 \item a general application sketch/scheme,
\item a description of every component implemented, at module level,
\item class diagrams, important classes and methods from key classes.
\end{itemize}

\chapter{Testing and Validation}

About 5\% of the paper
\section{Title}
\section{Other title}

\chapter{User's manual}

In the installation description section your should detail the hardware and software resources needed for installing and running the application, and a step by step description of how your application can be deployed/installed. An administrator should be able to perform the installation/deployment based on your instructions.

In the user manual section you describe how to use the application from the point of view of a user with no inside technical information; this should be done with screen shots and a stepwize explanation of the interaction. Based on user's manual, a person should be able to use your product.

\section{Title}
\section{Other title}

\chapter{Conclusions}

About. 5\% of the whole

Here your write:
\begin{itemize}
\item a summary of your contributions/achievements,
\item a critical analysis of the achieved results,
\item a description of the possibilities of improving/further development.
\end{itemize}
\section{Title}
\section{Other title}


%\addcontentsline {toc}{chapter}{Bibliography} 
\bibliographystyle{IEEEtran} 
\bibliography{thesis}%same file name as for .bib

\include{apendix}

\end{document}
